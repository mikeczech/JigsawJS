%
%  untitled
%
%  Created by Mike Czech on 2012-08-06.
%  Copyright (c) 2012 __MyCompanyName__. All rights reserved.
%
\documentclass[]{article}

% Use utf-8 encoding for foreign characters
\usepackage[utf8]{inputenc}

% Setup for fullpage use
\usepackage{fullpage}

% Uncomment some of the following if you use the features
%
% Running Headers and footers
%\usepackage{fancyhdr}

% Multipart figures
%\usepackage{subfigure}

% More symbols
%\usepackage{amsmath}
%\usepackage{amssymb}
%\usepackage{latexsym}
%\usepackage{enumerate}
%\setlength{\parindent}{0pt}

% Surround parts of graphics with box
\usepackage{boxedminipage}

% Package for including code in the document
\usepackage{listings}

% If you want to generate a toc for each chapter (use with book)
\usepackage{minitoc}

% This is now the recommended way for checking for PDFLaTeX:
\usepackage{ifpdf}

%\newif\ifpdf
%\ifx\pdfoutput\undefined
%\pdffalse % we are not running PDFLaTeX
%\else
%\pdfoutput=1 % we are running PDFLaTeX
%\pdftrue
%\fi
\usepackage{url}
\ifpdf
\usepackage[pdftex]{graphicx}
\else
\usepackage{graphicx}
\fi
\title{\textsc{JigsawJS} \\ Anleitung }
\author{ }

\date{\today}

\begin{document}

\ifpdf
\DeclareGraphicsExtensions{.pdf, .jpg, .tif}
\else
\DeclareGraphicsExtensions{.eps, .jpg}
\fi

\maketitle

\section{Voraussetzungen} % (fold)
\label{sub:voraussetzungen}

\begin{itemize}
  \item NodeJS (\url{http://nodejs.org/})
  \item z.B. XAMPP (\url{www.apachefriends.org/de/xampp.html})
\end{itemize}

Beschreibungen zur Installation sind auf den jeweiligen Internetpräsenzen
der Projekte vorhanden.

\section{Einrichtung} % (fold)
\label{sub:installation}

\begin{enumerate}
  \item Verschiebe die Projektdateien in ein Verzeichis, welches vom Webserver (XAMPP) öffentlich gemacht wird. Unter
  MacOSX ist dies z.B. /Users/\$USERNAME/Sites. 
  \item Starte nun ein Terminal und wechsel in das entpackte Verzeichnis.
  \item Navigiere zum Unterordner puzzles/client und öffne die Datei \emph{config.js} mit einem beliebigen Texteditor. Folgende Änderungen müssen durchgeführt werden:
  
  \begin{itemize}
    \item brokerAddress: Hier muss die lokale IP im Netzwerk mit Port 5050 angegeben werden.
    \item loginAddress: IP des Rechners auf dem die Login-Einheit laufen wird, Port 3000.
  \end{itemize}
  
  \item Öffne in dem aktuellen Verzeichnis die Datei index.html mit einem beliebigen Texteditor und ändere die IP der folgenden Zeile in
  deine lokale IP (mit Port 5050) um.
  \begin{verbatim}
    <script src="http://192.168.178.71:5050/socket.io/socket.io.js" 
    type="text/javascript" charset="utf-8"></script>
  \end{verbatim}
  
  \item Wechsel nun \textbf{relativ zum entpackten Verzeichnis} in den Ordner puzzles/server.
  \item Öffne die Datei \emph{config.js} und führe folgende Änderungen durch:
  
  \begin{itemize}
    \item mcastAddress, mcastPort: Adresse des Rechners auf dem die Multicast-Einheit gestartet wird.
    \item loginAddress, loginPort: Adresse des Rechners auf die die Login-Einheit gestartet wird.
    \item isLeader: True, falls der Broker Leader werden soll (siehe Ausarbeitung)
  \end{itemize}
  
  \item \textbf{Die folgenden Punkte müssen nur abgearbeitet werden, falls der aktuelle Rechner auch die Login-Einheit bzw.
    die Multicast-Einheit verwalten soll.}
  
  \item Öffne die Datei \emph{login.js} und passe den Pfad (Zeile 28) entsprechend zu dem entpackten Archiv an.
  
  \begin{verbatim}
    res.redirect("http://" + best + "/~mike/jigsawjs/code/codiqa-app/app.html");
  \end{verbatim}
  
  (Wenn das Archiv z.B. direkt in /Users/\$USERNAME/Sites entpackt wurde, muss nur 'mike' mit dem aktuellen
  Benutzernamen ausgetauscht werden)
  
  \item Weiterhin müssen in das Feld \emph{nodes} alle Rechner-IPs eingetragen werden, die einen aktiven Broker verwalten.
  Dies kann dann z.B. so aussehen:
  \begin{verbatim}
    var nodes = ['131.234.78.75', '192.168.123.12'];
  \end{verbatim}
  
\end{enumerate}

\section{Starten der Login-Einheit} % (fold)
\label{sec:starten_eines_brokers}

\begin{itemize}
  \item Wechsel, ausgehend vom Projektverzeichnis, nach puzzles/server
  \item Führe \emph{node login.js} im Terminal aus
\end{itemize}

% section starten_eines_brokers (end)

\section{Starten der Multicast-Einheit} % (fold)
\label{sec:starten_der_multicast_einheit}

\begin{itemize}
  \item Wechsel, ausgehend vom Projektverzeichnis, nach puzzles/server
  \item Führe \emph{node multicast.js} im Terminal aus
\end{itemize}

% section starten_der_multicast_einheit (end)

\section{Starten eines Brokers} % (fold)
\label{sec:starten_eines_brokers}

\begin{itemize}
  \item Starte den XAMPP Webserver
  \item Wechsel, ausgehend vom Projektverzeichnis, nach puzzles/server
  \item Führe \emph{node server.js} im Terminal aus
\end{itemize}

Um zu testen, ob die Einrichtung einwandfrei geklappt hat, kann in einem beliebigen Browser des Netzwerks
die Adresse der Login-Einheit mit Port 3000 aufgerufen werden. Diese führt dann bei Erfolg eine Weiterleitung zu
einem Broker durch, der den Client bereitstellt. Ich empfehle für einen ersten Testdurchlauf nur einen Broker zu verwenden. 
Zum Beispiel kann der Aufruf im Browser wie folgt aussehen:

\begin{verbatim}
  http://127.0.0.1:3000
\end{verbatim}

% section starten_eines_brokers (end)

% subsection installation (end)

% subsection subsection_name (end)

% subsection voraussetzungen (end)
% subsubsection vor (end)

% section installation (end)

\end{document}
